%%%%%%%%%%%%%%%%%%%%%%%%%%%%%%%%%%%%%%%%%%%
%
%		David Vorous
%		ECE-351-51
%		Lab #10
%		Due: Nov. 6, 2021
%
%%%%%%%%%%%%%%%%%%%%%%%%%%%%%%%%%%%%%%%%%%%

\documentclass[12pt]{report}
\usepackage[english]{babel}
%\usepackage{natbib}
\usepackage{url}
\usepackage[utf8x]{inputenc}
\usepackage{amsmath}
\usepackage{graphicx}
\graphicspath{ {./Graphics/} }
\usepackage{parskip}
\usepackage{fancyhdr}
\usepackage{vmargin}
\usepackage{listings}
\usepackage{hyperref}
\usepackage{xcolor}
\usepackage{steinmetz}

\definecolor{codegreen}{rgb}{0,0.6,0}
\definecolor{codegray}{rgb}{0.5,0.5,0.5}
\definecolor{codeblue}{rgb}{0,0,0.95}
\definecolor{backcolour}{rgb}{0.95,0.95,0.92}

\lstdefinestyle{mystyle}{
    backgroundcolor=\color{backcolour},   
    commentstyle=\color{codegreen},
    keywordstyle=\color{codeblue},
    numberstyle=\tiny\color{codegray},
    stringstyle=\color{codegreen},
    basicstyle=\ttfamily\footnotesize,
    breakatwhitespace=false,         
    breaklines=true,                 
    captionpos=b,                    
    keepspaces=true,                 
    numbers=left,                    
    numbersep=5pt,                  
    showspaces=false,                
    showstringspaces=false,
    showtabs=false,                  
    tabsize=2
}
 
\lstset{style=mystyle}

\setmarginsrb{3 cm}{2.5 cm}{3 cm}{2.5 cm}{1 cm}{1.5 cm}{1 cm}{1.5 cm}

\title{LAB 10}								
\author{ David Vorous}						
\date{11/6/21}

\makeatletter
\let\thetitle\@title
\let\theauthor\@author
\let\thedate\@date
\makeatother

\pagestyle{fancy}
\fancyhf{}
\rhead{\theauthor}
\lhead{\thetitle}
\cfoot{\thepage}

\begin{document}

\begin{titlepage}
	\centering
    \vspace*{0.5 cm}
\begin{center}    \textsc{\Large   ECE 351 - Section 51 }\\[2.0 cm]	\end{center}
	\textsc{\Large Frequency Response }\\[0.5 cm]
	\rule{\linewidth}{0.2 mm} \\[0.4 cm]
	{ \huge \bfseries \thetitle}\\
	\rule{\linewidth}{0.2 mm} \\[1.5 cm]
	
	\begin{minipage}{0.4\textwidth}
		\begin{flushleft} \large
			\end{flushleft}
			\end{minipage}~
			\begin{minipage}{0.4\textwidth}
            
			\begin{flushright} \large
			\emph{Submitted By :} \\
			David Vorous  
		\end{flushright}
           
	\end{minipage}\\[2 cm]

\end{titlepage}


\tableofcontents

\pagebreak

\renewcommand{\thesection}{\arabic{section}}

\section{Introduction}
Lab 10 introduces the use of Bode-Plots and the passing of signals through filters. It is known by now, that all simple periodic signals can be composed of varying frequencies added together. The Fourier Transform of such a simple function depicts all the frequencies and magnitudes necessary to construct it in the time domain, but it also gives us a glimpse into its interaction with a filter. A filter is a device (mathematical and physical) which subdues and amplifies certain frequencies. It is due to this frequency operation that we must consider input signals in the frequency domain. For this Lab, we ran multiple simple signals into a single filter, and plotted the results. The libraries used in the lab were numpy (imported as "np"), matplotlib (imported as "plt"), scipy.signal (imported as "sig"), and control (imported as "con").

\section{Equations}
Input function:
\begin{equation}
    x(t) = \text{cos}(2\pi \cdot 100t)+\text{cos}(2\pi \cdot 3024t)+\text{sin}(2\pi \cdot 50000t)
\end{equation}
Filter functions:
\begin{equation}
    \text{H}(s) = \frac{\frac{1}{RC}s}{s^{2}+(\frac{1}{RC})s+\frac{1}{LC}}
\end{equation}
\begin{equation}
    |\text{H}(j\omega)| = \frac{|\frac{\omega}{RC}|}{\sqrt{(\frac{1}{LC}-\omega^{2})^{2}+(\frac{\omega}{RC})^{2}}}
\end{equation}
\begin{equation}
    \phase{\text{ H}(j\omega)^{\circ}\text{ }} = 90^{\circ} - \text{atan2}\left(\frac{\omega}{RC},\frac{1}{LC}-\omega^{2}\right) ^{\circ}
\end{equation}

\pagebreak

\section{Methodology}
First, the input function and the Filter's polar-components were defined using the inherent Python tool-set and numpy library. Additionally, the con.bode() function was used with the numerator and denominator of the transfer-function to compare results. Next, the previously defined numerator and denominator were transformed into the z-domain, and passed into the sig.lfilter() function to find the frequency response due to the filter:

\begin{lstlisting}[language=Python]
steps1 = 1
steps2 = 3.183e-6
w = np.arange(1e3, 1e6 + steps1, steps1)
t = np.arange(0, 1e-2 + steps2, steps2)

r = 1e3
c = 100e-9
l = 27e-3

# --- User - Defined Functions ---
def mag_H(w, r, c, l):
    return (w/(r*c))/np.sqrt(((1/(l*c))-w**2)**2+(w/(r*c))**2)

def dB(x):
    return 20*np.log10(x)

def ang_H(w, r, c, l):
    return 90-np.arctan2((w/(r*c)),((1/(l*c))-w**2))*(180/np.pi)

def x(t):
    return np.cos(200*np.pi*t)+np.cos(6048*np.pi*t)+np.sin(10e5*np.pi*t)

Hnum = [0, 1/(r*c), 0]
Hden = [1, 1/(r*c), 1/(l*c)]
sys = con.TransferFunction(Hnum, Hden)
_ = con.bode(sys, w, dB = True, Hz = True, deg = True, Plot = True)

znum, zden = sig.bilinear(Hnum, Hden, 1/steps2)
filt = sig.lfilter(znum, zden, x(t))
# ---
\end{lstlisting}
Afterwards, the matplotlib.pyplot and control libraries were used to plot the relevant data. \pagebreak

\section{Plots}
\begin{center}
    \includegraphics[scale=0.55]{Figure 2021-11-07 135216 (2)}\\~\\~\\
    Magnitude and Phase Bode plots (Left: Hand-written, Right: con.bode()):\\~\\
    \includegraphics[scale=0.35]{Figure 2021-11-07 135216 (1)}
    \includegraphics[scale=0.5]{Figure 2021-11-07 135216 (0)}\\~\\~\\
    \includegraphics[scale=0.55]{Figure 2021-11-07 135216 (3)}

\end{center}

\section{Results}
The results of Lab 10 were partially expected. While the hand-written Bode plot and con.bode plot look the same, their units are not the same. Aside from this, the plots of $x(t)$ and its filtered equivalent make sense. The reason for why the filtered $x(t)$ makes sense will be answered in the Questions section.
\pagebreak

\section{Questions}
\begin{enumerate}
    \item{Explain how the filter and filtered output makes sense given the Bode plots. Discuss how the filter modifies specific frequency bands, in Hz.}\\~\\
    \textit{ It can be observed that x(t) is composed of high, medium, and low frequencies. According to the hand-written bode plot, the low frequency lands to the left of the 10 $^{3}$ rad/s mark, the medium frequency lands almost perfectly at the Bode plots peak, and the high frequency lands between 10 $^{5}$ and 10 $^{6}$ rad/s. From this, we know that all but the medium frequency must be filtered off... and that is indeed what we see in the filtered result.}\\
    \item Discuss the purpose and workings of scipy.signal.bilinear() and scipy.signal.lfilter().\\~\\
    \textit{The sig.bilinear() function transforms the Transfer function into the z-domain: the transformation is a way of representing the Transfer function in a digital space. Then, the sig.lfilter() function applies the z-domain filter to the time-domain $x(t)$. This latter function specifically requires digital filters of either IIR or FIR type.}\\
    \item What happens if you use a different sampling frequency in scipy.signal.bilinear() than used for the time-domain signal?\\~\\
    \textit{the sig.bilinear() function takes a separate argument for sampling frequency, which may differ from the time variable for the filtered function. This mismatch in sampling frequency will lead to the wrong frequencies being filtered-off, or other artifacts. If the sig.bilinear() frequency is too large, larger frequencies will emerge from the plot. Conversely, if it is too small, smaller frequencies will emerge. To ensure the desired frequencies are filtered, make sure the sampling frequency and the time variable agree.}
\end{enumerate}

\pagebreak

\section{Conclusion}
This Lab provided a decent introduction to Python-enabled Bode plots. Bode plots represent the characteristics of a filter, and are exceedingly important in signal processing and generation. While Bode plots are incredibly important, I personally find it annoying to convert between dB and magnitude, Hz and rad/s, etc... my assumption is that such conversions can be effectively approximated given enough experience. Aside from this inconvenience, all the library functions and plots made sense, and the Lab was enjoyable to complete.

All files can be found on my GitHub Repositories tab:\\
\url{https://github.com/David-Vorous?tab=repositories}

\end{document}