%%%%%%%%%%%%%%%%%%%%%%%%%%%%%%%%%%%%%%%%%%%
%
%		David Vorous
%		ECE-351-51
%		Lab #4
%		Due: Sep. 28, 2021
%
%%%%%%%%%%%%%%%%%%%%%%%%%%%%%%%%%%%%%%%%%%%

\documentclass[12pt]{report}
\usepackage[english]{babel}
%\usepackage{natbib}
\usepackage{url}
\usepackage[utf8x]{inputenc}
\usepackage{amsmath}
\usepackage{graphicx}
\graphicspath{ {./Graphics/} }
\usepackage{parskip}
\usepackage{fancyhdr}
\usepackage{vmargin}
\usepackage{listings}
\usepackage{hyperref}
\usepackage{xcolor}

\definecolor{codegreen}{rgb}{0,0.6,0}
\definecolor{codegray}{rgb}{0.5,0.5,0.5}
\definecolor{codeblue}{rgb}{0,0,0.95}
\definecolor{backcolour}{rgb}{0.95,0.95,0.92}

\lstdefinestyle{mystyle}{
    backgroundcolor=\color{backcolour},   
    commentstyle=\color{codegreen},
    keywordstyle=\color{codeblue},
    numberstyle=\tiny\color{codegray},
    stringstyle=\color{codegreen},
    basicstyle=\ttfamily\footnotesize,
    breakatwhitespace=false,         
    breaklines=true,                 
    captionpos=b,                    
    keepspaces=true,                 
    numbers=left,                    
    numbersep=5pt,                  
    showspaces=false,                
    showstringspaces=false,
    showtabs=false,                  
    tabsize=2
}
 
\lstset{style=mystyle}

\setmarginsrb{3 cm}{2.5 cm}{3 cm}{2.5 cm}{1 cm}{1.5 cm}{1 cm}{1.5 cm}

\title{LAB 4}								
\author{ David Vorous}						
\date{9/28/21}

\makeatletter
\let\thetitle\@title
\let\theauthor\@author
\let\thedate\@date
\makeatother

\pagestyle{fancy}
\fancyhf{}
\rhead{\theauthor}
\lhead{\thetitle}
\cfoot{\thepage}

\begin{document}

\begin{titlepage}
	\centering
    \vspace*{0.5 cm}
\begin{center}    \textsc{\Large   ECE 351 - Section 51 }\\[2.0 cm]	\end{center}
	\textsc{\Large System Step Response Using Convolution  }\\[0.5 cm]
	\rule{\linewidth}{0.2 mm} \\[0.4 cm]
	{ \huge \bfseries \thetitle}\\
	\rule{\linewidth}{0.2 mm} \\[1.5 cm]
	
	\begin{minipage}{0.4\textwidth}
		\begin{flushleft} \large
			\end{flushleft}
			\end{minipage}~
			\begin{minipage}{0.4\textwidth}
            
			\begin{flushright} \large
			\emph{Submitted By :} \\
			David Vorous  
		\end{flushright}
           
	\end{minipage}\\[2 cm]

\end{titlepage}


\tableofcontents

\pagebreak

\renewcommand{\thesection}{\arabic{section}}

\section{Introduction}

Lab 4 is meant to better acquaint the students with the library convolve() function. After creating a handful of of simple input functions, the students were tasked with hand-calculating their step response (convolution with u(t)). The plots for these hand-calculated convolutions were then compared with the plots for the library convolve() alternatives. The libraries used in the lab were numpy (imported as "np"), matplotlib (imported as "plt"), and scipy.signals (imported as "sig").

\section{Equations}

The first equations used in the Lab were simple:
\begin{equation}
    \text{h1}(t) =e^{-2t} \cdot (u(t) - u(t - 3))
\end{equation}
\begin{equation}
    \text{h2}(t) = u(t - 2) - u(t - 6)
\end{equation}
\begin{equation}
    \text{h3}(t) = \text{cos}(4t) \cdot u(t)
\end{equation}
Then, we calculated their step response:
\begin{equation}
    (\text{u*h1})(t) = \frac{1}{2} [(1 - e^{-2t}) \cdot u(t) - (1 - e^{-2(t - 3)}) \cdot u(t - 3)]
\end{equation}
\begin{equation}
    (\text{u*h2})(t) = r(t - 2) - r(t - 6)
\end{equation}
\begin{equation}
    (\text{u*h3})(t) = \frac{1}{4} \cdot \text{sin}(4t) \cdot u(t)
\end{equation}

Where $u(t)$ is the Step Function and $r(t)$ is the Ramp Function.
\pagebreak


\section{Methodology}

Both the step and ramp functions utilized incremental sampling to produce the necessary output arrays. Each hX was built using u(t), r(t), and some library trig and exponential functions. h1 is a decaying exponential which starts at t = 0, h2 is a block of height 1 and width 4, and h3 is a cosine wave starting at t = 0. The step response for each function was also defined:
\begin{lstlisting}[language=Python]
def h1(t):
    return np.exp(-2 * t) * (u(t) - u(t - 3))

def h2(t):
    return u(t-2) - u(t-6)

def h3(t):
    return np.cos(4 * t) * u(t)

def step_h1(t):
    return 0.5*(1-np.exp(-2*t))*u(t) - 0.5*(1-np.exp(-2*(t-3)))*u(t-3)

def step_h2(t):
    return r(t-2) - r(t-6)

def step_h3(t):
    return 0.25 * np.sin(4*t)*u(t)
\end{lstlisting}
Afterwards, the matplotlib.pyplot library was used to plot the simple functions, and their convolutions.


\section{Plots}
\begin{center}
    \includegraphics[scale=0.4]{Figure 2021-09-28 140144 (0)}
\end{center}
\text{}
\begin{center}
    \includegraphics[scale=0.333]{Figure 2021-09-28 140144 (1)}
    \includegraphics[scale=0.333]{Figure 2021-09-28 140144 (2)}
    \includegraphics[scale=0.333]{Figure 2021-09-28 140144 (3)}
\end{center}

\section{Results}

The results of the Lab were actually quite unexpected. The library calculated convolutions did not quite match their hand-calculated counterparts. (u*h2) and (u*h3) are slightly exempt from this critique, due to the domain differences between u(t) and hx(t); but (u*h1) differs prior to that domain conflict. It's as if the convolve library function fails to notice the effect of u(t - 3) in h1.

\section{Conclusion}

This Lab certainly raised some questions regarding the validity of the library convolve function. It appears to work well within certain regions, but less so in others; my assumption is that these errors are consistent and therefore avoidable. Aside from the library convolution plots, the Lab went rather smoothly; the plots of the basic functions and their hand-calculated step responses were consistent with expectations.
The Lab was clear in it's expectations, but perhaps a hint that our hand-calculated convolutions will differ from the library convolutions will alleviate some confusion.\\

All files can be found on my GitHub Repositories tab:\\
\url{https://github.com/David-Vorous?tab=repositories}

\end{document}