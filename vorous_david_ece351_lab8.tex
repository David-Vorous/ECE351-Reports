%%%%%%%%%%%%%%%%%%%%%%%%%%%%%%%%%%%%%%%%%%%
%
%		David Vorous
%		ECE-351-51
%		Lab #8
%		Due: Oct. 26, 2021
%
%%%%%%%%%%%%%%%%%%%%%%%%%%%%%%%%%%%%%%%%%%%

\documentclass[12pt]{report}
\usepackage[english]{babel}
%\usepackage{natbib}
\usepackage{url}
\usepackage[utf8x]{inputenc}
\usepackage{amsmath}
\usepackage{graphicx}
\graphicspath{ {./Graphics/} }
\usepackage{parskip}
\usepackage{fancyhdr}
\usepackage{vmargin}
\usepackage{listings}
\usepackage{hyperref}
\usepackage{xcolor}

\definecolor{codegreen}{rgb}{0,0.6,0}
\definecolor{codegray}{rgb}{0.5,0.5,0.5}
\definecolor{codeblue}{rgb}{0,0,0.95}
\definecolor{backcolour}{rgb}{0.95,0.95,0.92}

\lstdefinestyle{mystyle}{
    backgroundcolor=\color{backcolour},   
    commentstyle=\color{codegreen},
    keywordstyle=\color{codeblue},
    numberstyle=\tiny\color{codegray},
    stringstyle=\color{codegreen},
    basicstyle=\ttfamily\footnotesize,
    breakatwhitespace=false,         
    breaklines=true,                 
    captionpos=b,                    
    keepspaces=true,                 
    numbers=left,                    
    numbersep=5pt,                  
    showspaces=false,                
    showstringspaces=false,
    showtabs=false,                  
    tabsize=2
}
 
\lstset{style=mystyle}

\setmarginsrb{3 cm}{2.5 cm}{3 cm}{2.5 cm}{1 cm}{1.5 cm}{1 cm}{1.5 cm}

\title{LAB 8}								
\author{ David Vorous}						
\date{10/26/21}

\makeatletter
\let\thetitle\@title
\let\theauthor\@author
\let\thedate\@date
\makeatother

\pagestyle{fancy}
\fancyhf{}
\rhead{\theauthor}
\lhead{\thetitle}
\cfoot{\thepage}

\begin{document}

\begin{titlepage}
	\centering
    \vspace*{0.5 cm}
\begin{center}    \textsc{\Large   ECE 351 - Section 51 }\\[2.0 cm]	\end{center}
	\textsc{\Large Fourier Series Approximation of a Square Wave }\\[0.5 cm]
	\rule{\linewidth}{0.2 mm} \\[0.4 cm]
	{ \huge \bfseries \thetitle}\\
	\rule{\linewidth}{0.2 mm} \\[1.5 cm]
	
	\begin{minipage}{0.4\textwidth}
		\begin{flushleft} \large
			\end{flushleft}
			\end{minipage}~
			\begin{minipage}{0.4\textwidth}
            
			\begin{flushright} \large
			\emph{Submitted By :} \\
			David Vorous  
		\end{flushright}
           
	\end{minipage}\\[2 cm]

\end{titlepage}


\tableofcontents

\pagebreak

\renewcommand{\thesection}{\arabic{section}}

\section{Introduction}

Lab 8 introduces the students to Fourier Series representation. Representing a given periodic function in terms of added frequencies is incredibly important in signal processing, probability, and the "heat equation." The "heat equation" describes the propagation of \textit{heat} through an n-dimensional body. Joseph Fourier actually developed his Fourier series algorithm in an attempt to solve the heat equation. For the purposes of this Lab however, we care less about application, and more about generation. In that spirit, the students were tasked with generating an additive, multi-variate series which approximates a simple square wave, \textit{x(t)}. The libraries used in the lab were numpy (imported as "np") and matplotlib (imported as "plt").

\begin{center}
    \includegraphics[scale=0.3]{Input}\\
    \text{Figure 1: \textit{x(t)} for ECE 351 Lab 8}
\end{center}

\section{Equations}
General Fourier Series Setup:
\begin{equation}
    x(t) = \frac{1}{2} a_0 + \sum_{k=1}^{\infty} a_k\text{cos}(k\omega_0t) + b_k\text{sin}(k\omega_0t)
\end{equation}
Where,\\
\begin{equation}
    a_k = \frac{2}{T} \int_{0}^{T} x(t)\text{cos}(k\omega_0t) \,dt = 0
\end{equation}
\begin{equation}
    b_k = \frac{2}{T} \int_{0}^{T} x(t)\text{sin}(k\omega_0t) \,dt = \frac{2\cdot(1 - \text{cos}(k\pi))}{k\pi}
\end{equation}
\begin{equation}
    \omega_0 = \frac{2\pi}{T}
\end{equation}
Therefore, the series equation is:
\begin{equation}
    x(t,T) = \frac{2}{\pi}\cdot \sum_{k=1}^{\infty} \frac{(1 - \text{cos}(k\pi))}{k}\cdot\text{sin}\left(\frac{2\pi kt}{T}\right)
\end{equation}

\pagebreak

\section{Methodology}
Defining $a_k$, $b_k$, and $\omega_0$ was done prior to the Lab, and their code was achieved using simple numpy constants/functions. For the series approximation of $x(t)$, a for-loop along with numpy.sin() was used, while also introducing the parameters $T$ and $N$:

\begin{lstlisting}[language=Python]
def w(T):
    return ((2 * np.pi)/T)
def bk(k):
    return ((-2.0/(k * np.pi)) * (np.cos(k * np.pi) - 1))

print (bk(1), bk(2), bk(3))

def x(t, T, N):
    y = 0
    for k in np.arange(1, N + 1):
        y += ((np.sin(k * w(T) * t)) * bk(k))
    return y
\end{lstlisting}
Afterwards, the matplotlib.pyplot library was used to plot $x(t)$ with T = 8.


\section{Plots}

\begin{center}
    \includegraphics[scale=0.4]{Figure 2021-10-26 154340 (0)}\\
    Figure 2: $x(t)$ with increasing summation upper bounds\\~\\
\end{center}
\begin{center}
    \includegraphics[scale=0.39]{Figure 2021-10-26 154340 (1)}\\
    Figure 3: $x(t)$ with all previous graphs overlapped
\end{center}

\section{Results}
The results of Lab 8 needed to conform to very specific requirements, making testing relatively easy; we knew exactly what our summation was meant to approach, so poor code would become immediately apparent after running the program. Luckily, all went well, and the plots showed exactly the behavior they should have, as confirmed by similar entry into an alternative online graphing resource.

\section{Questions}
\begin{enumerate}
    \item \textit{Is x(t) an even or an odd function? Explain why.}
    \begin{center}
    $x(t)$ is an odd function: $x(t) \neq x(-t)$, $x(t) = -x(-t)$.
    \end{center}
    \item \textit{Based on your results from Task 1, what do you expect the values of \\$a_2$, $a_3$, . . . , $a_n$ to be?
Why?}
    \begin{center}
    $a_k = 0$, $b_1 = \frac{4}{\pi}$, $b_2 = 0$, $b_3 = \frac{4}{3\pi}$, $b_4 = 0$, ...\\
    \end{center}
    From solving $a_k$ and $b_k$ in terms of integer inputs, a simple pattern emerges for both. Especially in the case of $a_k$, which is 0 for all $k$ inputs.
    \item \textit{How does the approximation of the square wave change as the value of N increases? In what
way does the Fourier series struggle to approximate the square wave?}\\\\
    As $N$ increases, more frequencies are thrown-in to refine and improve the approximation. However, no frequencies can handle perfect vertical jumps, as is present in the square wave. It would require an infinite number of frequencies to properly hug these discontinuities.
    \item \textit{What is occurring mathematically in the Fourier series summation as the value of N increases?}\\\\
    As $N$ increases, waves are added which behave destructively in some regions, and constructively in others. In the case of the square wave, each successive wave is destructive at the regions where $x(t)$ is flat, and constructive where $x(t)$ has a discontinuity. This interplay between waves allows for shaping which eventually resembles the original function.
\end{enumerate}

\pagebreak

\section{Conclusion}
This Lab provided excellent confirmation of the topics discussed in lecture. Fourier Series are a fairly strange idea to embrace, especially when they're extended to n-dimensions, and allowed to trace-out complicated n-dimensional shapes. The square wave example in the Lab proved to be a useful benchmark for the power and limitations of the Fourier Transform, especially around discontinuities. It was nice to plot the same sum for different upper bounds, to see how the series progresses. Aside from some difficulties with coding the summation, the Lab went smoothly, and was enjoyable to complete.

All files can be found on my GitHub Repositories tab:\\
\url{https://github.com/David-Vorous?tab=repositories}

\end{document}