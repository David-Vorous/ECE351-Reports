%%%%%%%%%%%%%%%%%%%%%%%%%%%%%%%%%%%%%%%%%%%
%
%		David Vorous
%		ECE-351-51
%		Lab #9
%		Due: Nov. 1, 2021
%
%%%%%%%%%%%%%%%%%%%%%%%%%%%%%%%%%%%%%%%%%%%

\documentclass[12pt]{report}
\usepackage[english]{babel}
%\usepackage{natbib}
\usepackage{url}
\usepackage[utf8x]{inputenc}
\usepackage{amsmath}
\usepackage{graphicx}
\graphicspath{ {./Graphics/} }
\usepackage{parskip}
\usepackage{fancyhdr}
\usepackage{vmargin}
\usepackage{listings}
\usepackage{hyperref}
\usepackage{xcolor}

\definecolor{codegreen}{rgb}{0,0.6,0}
\definecolor{codegray}{rgb}{0.5,0.5,0.5}
\definecolor{codeblue}{rgb}{0,0,0.95}
\definecolor{backcolour}{rgb}{0.95,0.95,0.92}

\lstdefinestyle{mystyle}{
    backgroundcolor=\color{backcolour},   
    commentstyle=\color{codegreen},
    keywordstyle=\color{codeblue},
    numberstyle=\tiny\color{codegray},
    stringstyle=\color{codegreen},
    basicstyle=\ttfamily\footnotesize,
    breakatwhitespace=false,         
    breaklines=true,                 
    captionpos=b,                    
    keepspaces=true,                 
    numbers=left,                    
    numbersep=5pt,                  
    showspaces=false,                
    showstringspaces=false,
    showtabs=false,                  
    tabsize=2
}
 
\lstset{style=mystyle}

\setmarginsrb{3 cm}{2.5 cm}{3 cm}{2.5 cm}{1 cm}{1.5 cm}{1 cm}{1.5 cm}

\title{LAB 9}								
\author{ David Vorous}						
\date{11/1/21}

\makeatletter
\let\thetitle\@title
\let\theauthor\@author
\let\thedate\@date
\makeatother

\pagestyle{fancy}
\fancyhf{}
\rhead{\theauthor}
\lhead{\thetitle}
\cfoot{\thepage}

\begin{document}

\begin{titlepage}
	\centering
    \vspace*{0.5 cm}
\begin{center}    \textsc{\Large   ECE 351 - Section 51 }\\[2.0 cm]	\end{center}
	\textsc{\Large Fast Fourier Transform (FFT) }\\[0.5 cm]
	\rule{\linewidth}{0.2 mm} \\[0.4 cm]
	{ \huge \bfseries \thetitle}\\
	\rule{\linewidth}{0.2 mm} \\[1.5 cm]
	
	\begin{minipage}{0.4\textwidth}
		\begin{flushleft} \large
			\end{flushleft}
			\end{minipage}~
			\begin{minipage}{0.4\textwidth}
            
			\begin{flushright} \large
			\emph{Submitted By :} \\
			David Vorous  
		\end{flushright}
           
	\end{minipage}\\[2 cm]

\end{titlepage}


\tableofcontents

\pagebreak

\renewcommand{\thesection}{\arabic{section}}

\section{Introduction}

Lab 9 introduces the students to a computational method which approximates the Fourier transform of some input. The method, called the Fast Fourier Transform (FFT), works by taking discrete samples from an input (usually as data in an array) and decomposing the trend into simple frequencies. Due to the discrete nature of the FFT, continuous behavior is achieved by simply increasing the input resolution (input array density). For this Lab, we looked at the FFT for four linear combinations of sinusoids, one of which approximates a square wave. Additionally, the students were to implement a de-noising algorithm to make the resulting FFT phase plots more readable. The libraries used in the lab were numpy (imported as "np"), matplotlib (imported as "plt"), and scipy.fftpack.

\section{Equations}
FFT Inputs:
\begin{equation}
    x_{1}(t) = \text{cos}(2\pi t)
\end{equation}
\begin{equation}
    x_{2}(t) = 5\cdot \text{sin}(2\pi t)
\end{equation}
\begin{equation}
    x_{3}(t) = 2\cdot \text{cos}(4\pi t - 2) + \text{sin}^{2}(12\pi t + 3)
\end{equation}
\begin{equation}
    x_{4}(t) = \frac{2}{\pi}\cdot \sum_{k=1}^{15} \left(\frac{1 - \text{cos}(k\pi)}{k}\cdot\text{sin}\left(\frac{k\pi}{4}t\right)\right)
\end{equation}

\pagebreak

\section{Methodology}
Setting up the input functions was trivial, and implementing the provided FFT algorithm was also quite simple. Much of the Lab revolved around tweaking parameters and generating the necessary plots. Luckily, the input $x_4 (t)$ was generated from the previous Lab, and needed little modification to function independent of it's previous arguments:

\begin{lstlisting}[language=Python]
x1 = np.cos(2*np.pi*t1)
x2 = 5 * np.sin(2*np.pi*t1)
x3 = 2 * np.cos((2*np.pi*2*t1)-2) + (np.sin((2*np.pi*6*t1)+3))**2
def w(T):
    return ((2*np.pi)/T)
def bk(k):
    return ((2.0/(k*np.pi)) * (1-np.cos(k*np.pi)))
def x(t, T, N):
    y = 0
    for k in np.arange(1, N+1):
        y += ((np.sin(k*w(T)*t)) * bk(k))
    return y
x4 = x(t2, 8, 15)

def FFT1(x,fs):
    N = len(x)                        # find the length of the signal
    X_fft = scipy.fftpack.fft(x)      # perform the fast Fourier transform (fft)
    X_fft_shifted = scipy.fftpack.fftshift( X_fft ) # shift zero frequency components
                                                    # to the center of the spectrum             
    freq = np.arange(-N/2, N/2)*fs/N  # compute the frequencies for the output
                                      # signal , (fs is the sampling frequency and
                                      # needs to be defined previously in your code
    X_mag = np.abs( X_fft_shifted )/N # compute the magnitudes of the signal
    X_phi = np.angle( X_fft_shifted ) # compute the phases of the signal 
    return freq, X_mag, X_phi






def FFT2(x,fs):
    N = len(x)                        # find the length of the signal
    X_fft = scipy.fftpack.fft(x)      # perform the fast Fourier transform (fft)
    X_fft_shifted = scipy.fftpack.fftshift( X_fft ) # shift zero frequency components
                                                    # to the center of the spectrum             
    freq = np.arange(-N/2, N/2)*fs/N  # compute the frequencies for the output
                                      # signal , (fs is the sampling frequency and
                                      # needs to be defined previously in your code
    X_mag = np.abs( X_fft_shifted )/N # compute the magnitudes of the signal
    X_phi = np.angle( X_fft_shifted ) # compute the phases of the signal    
    for i in range(len(X_phi)):
        if (X_mag[i] < 1e-10):
            X_phi[i] = 0    
    return freq, X_mag, X_phi
\end{lstlisting}
Afterwards, the matplotlib.pyplot library was used to plot the FFT for each $x_i (t)$. \pagebreak


\section{Plots}
\begin{center}
    $x_{1}(t)$ Without De-Noising\\
    \includegraphics[scale=0.5]{Figure 2021-11-02 153133 (0)}\\~\\
    $x_{2}(t)$ Without De-Noising\\
    \includegraphics[scale=0.5]{Figure 2021-11-02 153133 (1)}
    \pagebreak~\\
    $x_{3}(t)$ Without De-Noising\\
    \includegraphics[scale=0.5]{Figure 2021-11-02 153133 (2)}\\~\\
    $x_{4}(t)$ Without De-Noising\\
    \includegraphics[scale=0.5]{Figure 2021-11-02 153133 (3)}
    \pagebreak~\\
\end{center}
\begin{center}
    $x_{1}(t)$ With De-Noising\\
    \includegraphics[scale=0.5]{Figure 2021-11-02 153133 (4)}\\~\\
    $x_{2}(t)$ With De-Noising\\
    \includegraphics[scale=0.5]{Figure 2021-11-02 153133 (5)}
    \pagebreak~\\
    $x_{3}(t)$ With De-Noising\\
    \includegraphics[scale=0.5]{Figure 2021-11-02 153133 (6)}\\~\\
    $x_{4}(t)$ With De-Noising\\
    \includegraphics[scale=0.5]{Figure 2021-11-02 153133 (7)}
    \pagebreak~\\
\end{center}

\section{Results}
The results of Lab 9 were slightly unexpected. The results show how discrete frequency checks (yielding $\delta (t - t_0)$s) can approach a continuous Fourier Transform given enough input function resolution. The de-noising algorithm worked like a charm, and all the plots, except for one, came out nicely. Regarding that one troublesome plot, the bottom-right plot of "$x_{1}(t)$ Without De-Noising" has an incorrect entry for $f = 0$. In reality, this value should be positive.

\section{Questions}
\begin{enumerate}
    \item What happens if sampling frequency is lower? If it is higher?\\~\\
    \textit{For a decent FFT, the sampling frequency must be high enough to pick-up the important details found in the input. These details influence the detected frequencies, and thus the FFT output. If the sampling frequency is too low, some frequencies are missed. If it's higher, more frequencies are picked up.}\\
    \item What difference does eliminating the small phase magnitudes make?\\~\\
    \textit{Let's consider vectors: a vector is a magnitude and an angle. If you plot a vectors magnitude and angle separately, you'll find that the angle is indiscriminate of the magnitude. Now consider a set of vectors, where a subset has magnitudes significantly smaller than the rest. It may be frustrating to find that this subset of tiny contributors... still contributes unobstructed angle data. To avoid this, we can filter off the subset entirely, ridding ourselves of their magnitude AND angle contributions. Doing so clears-up the Angle plot for our set of vectors. In our case, we are dealing with complex Fourier Transforms, which also have magnitudes and angles. But the same principle applies; if the angular data is cluttered, filter off the angles contributed by small magnitudes.}\\
    \item Verify your results from Tasks 1 and 2 using the Fourier transforms of cosine and sine. Explain why your results are correct. You will need the transforms in terms of Hz, not rad/s.\\~\\
    \textit{Considering the Fourier Transforms of sine and cosine, and their inclusion of $\delta (f \pm f_0)$s, it's safe to say that my results are accurate. Granted, while the plots returned from the FFT are only the magnitudes of the $\delta (f \pm f_0)$ functions (they do not reach to infinity), the placement of the spikes are accurate.}\\
\end{enumerate}

\section{Conclusion}
This Lab provided an excellent introduction to Fast Fourier Transform. I'd heard about this approach quite some time ago, but either didn't understand Fourier Transforms well enough at the time, or simply didn't grasp how the FFT worked. It is amazing to see how stable and deductive it is with a relatively small sampling frequency. Considering the property of duality for Fourier Transforms, I wonder if the IFFT (Inverse Fast Fourier Transform) uses a similar approach to undo the FFT. Aside from the error at $f(0)$ for $x_1 (t)$ in the noisy plot, the Lab went smoothly, and was enjoyable to complete.

All files can be found on my GitHub Repositories tab:\\
\url{https://github.com/David-Vorous?tab=repositories}

\end{document}