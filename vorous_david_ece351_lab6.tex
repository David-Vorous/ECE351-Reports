%%%%%%%%%%%%%%%%%%%%%%%%%%%%%%%%%%%%%%%%%%%
%
%		David Vorous
%		ECE-351-51
%		Lab #6
%		Due: Oct. 12, 2021
%
%%%%%%%%%%%%%%%%%%%%%%%%%%%%%%%%%%%%%%%%%%%

\documentclass[12pt]{report}
\usepackage[english]{babel}
%\usepackage{natbib}
\usepackage{url}
\usepackage[utf8x]{inputenc}
\usepackage{amsmath}
\usepackage{graphicx}
\graphicspath{ {./Graphics/} }
\usepackage{parskip}
\usepackage{fancyhdr}
\usepackage{vmargin}
\usepackage{listings}
\usepackage{hyperref}
\usepackage{xcolor}

\definecolor{codegreen}{rgb}{0,0.6,0}
\definecolor{codegray}{rgb}{0.5,0.5,0.5}
\definecolor{codeblue}{rgb}{0,0,0.95}
\definecolor{backcolour}{rgb}{0.95,0.95,0.92}

\lstdefinestyle{mystyle}{
    backgroundcolor=\color{backcolour},   
    commentstyle=\color{codegreen},
    keywordstyle=\color{codeblue},
    numberstyle=\tiny\color{codegray},
    stringstyle=\color{codegreen},
    basicstyle=\ttfamily\footnotesize,
    breakatwhitespace=false,         
    breaklines=true,                 
    captionpos=b,                    
    keepspaces=true,                 
    numbers=left,                    
    numbersep=5pt,                  
    showspaces=false,                
    showstringspaces=false,
    showtabs=false,                  
    tabsize=2
}
 
\lstset{style=mystyle}

\setmarginsrb{3 cm}{2.5 cm}{3 cm}{2.5 cm}{1 cm}{1.5 cm}{1 cm}{1.5 cm}

\title{LAB 6}								
\author{ David Vorous}						
\date{10/05/21}

\makeatletter
\let\thetitle\@title
\let\theauthor\@author
\let\thedate\@date
\makeatother

\pagestyle{fancy}
\fancyhf{}
\rhead{\theauthor}
\lhead{\thetitle}
\cfoot{\thepage}

\begin{document}

\begin{titlepage}
	\centering
    \vspace*{0.5 cm}
\begin{center}    \textsc{\Large   ECE 351 - Section 51 }\\[2.0 cm]	\end{center}
	\textsc{\Large Partial Fraction Decomposition  }\\[0.5 cm]
	\rule{\linewidth}{0.2 mm} \\[0.4 cm]
	{ \huge \bfseries \thetitle}\\
	\rule{\linewidth}{0.2 mm} \\[1.5 cm]
	
	\begin{minipage}{0.4\textwidth}
		\begin{flushleft} \large
			\end{flushleft}
			\end{minipage}~
			\begin{minipage}{0.4\textwidth}
            
			\begin{flushright} \large
			\emph{Submitted By :} \\
			David Vorous  
		\end{flushright}
           
	\end{minipage}\\[2 cm]

\end{titlepage}


\tableofcontents

\pagebreak

\renewcommand{\thesection}{\arabic{section}}

\section{Introduction}

Lab 6 is meant to introduce the students to partial fraction decomposition and the residue of a function. It is difficult to form a generalized algorithm for partial fraction decomposition using simple case-by-case rules; alternatively, the use of complex residues delivers all the coefficients for the expanded fraction set which, when combined with the poles and their degrees, results in the full expansion. In the case of Inverse Laplace Transforms, which heavily lean on partial fraction decomposition, rewriting the residues and poles into their expanded form is unnecessary, as these values can be plugged directly into certain inverse-Laplace formulas. The formula used in this lab follows an algorithm called the "sine method," which helps take the inverse Laplace of functions with simple complex poles. In this Lab, we were tasked with using residues in an inverse Laplace algorithm, and compare it with the included library step-response functions. The libraries used in the lab were numpy (imported as "np"), matplotlib (imported as "plt"), and scipy.signals (imported as "sig").

\section{Equations}

System A as a differential equation and its Transfer Function:
\begin{equation}
    y''(t) + 10y'(t)+24y(t)=x''(t)+6x'(t)+12x(t)
\end{equation}
\begin{equation}
    \text{H}(s) = \frac{s^{2} + 6s+12}{s^{2}+10s+24}
\end{equation}
System A Step Response:
\begin{equation}
    \text{Y}(s) = \frac{1}{2s} - \frac{1}{2(s+4)} + \frac{1}{s+6}
\end{equation}
\begin{equation}
    y(t) = \left(\frac{1}{2} - \frac{1}{2}e^{-4t} + e^{-6t}\right) \cdot u(t)
\end{equation}
System B as a differential equation and its Transfer Function:
\begin{equation}
    y^{(5)}(t)+18y^{(4)}(t)+218y^{(3)}(t)+2036y''(t)+9085y'(t)+25250y(t)=25250x(t)
\end{equation}
\begin{equation}
    \text{H}(s) = \frac{25250}{s^{5}+18s^{4}+218s^{3}+2036s^{2}+9085s+25250}
\end{equation}

Where $u(t)$ is the Unit Step Function.
\pagebreak


\section{Methodology}
As with so many of the Labs before, first defined was the step function (not shown). Then, the residue algorithm was created with the help of the inverse Laplace "sine method." Both of the Transfer functions for system A and system B were inputted into this algorithm, and compared to their mathematically equivalent counterparts. Namely, a hand-calculated time-domain step-response for system A, and a library-defined time-domain step-response for system B:

\begin{lstlisting}[language=Python]
def invLap(num, den, t):
    r,p,k = sig.residue(num, den)
    
    Combine = 0
    for i in range(len(r)):
        magn = np.abs(r[i])
        angl = np.angle(r[i])
        alph = np.real(p[i])
        omeg = np.imag(p[i])
        Combine += (magn * np.exp(alph * t) * np.cos((omeg * t) + angl)) * u(t) 
    return Combine

def reg_y(t):
    return (0.5 - 0.5 * np.exp(-4.0 * t) + np.exp(-6.0 * t)) * u(t)

hs = ([1, 6, 12],[1, 10, 24])

stept, stepy = sig.step(hs, T = t1)

ys1_num = [0, 1,  6, 12]
ys1_den = [1, 10, 24, 0]

r1,p1,k1 = sig.residue(ys1_num, ys1_den)

ys2_num = [0,0, 0,  0,   0,   0,25250]
ys2_den = [1,18,218,2036,9085,25250,0]

yt2 = invLap(ys2_num, ys2_den, t2)

ys3_num = [0,0, 0,  0,   0,   25250]
ys3_den = [1,18,218,2036,9085,25250]

libt, liby = sig.step((ys3_num,ys3_den), T = t2)
\end{lstlisting}
Afterwards, the matplotlib.pyplot library was used to plot all relevant data.


\section{Plots}
Left-most plots are System A, right-most are System B:
\begin{center}
    \includegraphics[scale=0.3]{Figure 2021-10-12 161901 (0)}
    \includegraphics[scale=0.3]{Figure 2021-10-12 161901 (1)}
\end{center}

\section{Results}
The results of Lab 6 confirmed the effectiveness of residues and the "sine method." When solving for the time-domain step-response of System A, neither residues nor the sine method were necessary yet... the algorithm yielded the same result. Then, the algorithm also yielded the same result as the sig.step() function included in the scipy.signals library. All expectations were therefore met.

\section{Conclusion}
This Lab provided satisfying reassurance of the methods discussed in lecture, as well as my own understanding of complex residues. Partial fraction decomposition is a deceptively complicated topic which initially appears to be simple algebra; only when delving into how it's calculated will the truth be revealed. Having employed the necessary techniques and avoided the occasional road-bumps, the Lab went very smooth. The Lab was clear in it's expectations, and was enjoyable to complete.\\

All files can be found on my GitHub Repositories tab:\\
\url{https://github.com/David-Vorous?tab=repositories}

\end{document}