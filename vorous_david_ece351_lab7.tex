%%%%%%%%%%%%%%%%%%%%%%%%%%%%%%%%%%%%%%%%%%%
%
%		David Vorous
%		ECE-351-51
%		Lab #7
%		Due: Oct. 19, 2021
%
%%%%%%%%%%%%%%%%%%%%%%%%%%%%%%%%%%%%%%%%%%%

\documentclass[12pt]{report}
\usepackage[english]{babel}
%\usepackage{natbib}
\usepackage{url}
\usepackage[utf8x]{inputenc}
\usepackage{amsmath}
\usepackage{graphicx}
\graphicspath{ {./Graphics/} }
\usepackage{parskip}
\usepackage{fancyhdr}
\usepackage{vmargin}
\usepackage{listings}
\usepackage{hyperref}
\usepackage{xcolor}

\definecolor{codegreen}{rgb}{0,0.6,0}
\definecolor{codegray}{rgb}{0.5,0.5,0.5}
\definecolor{codeblue}{rgb}{0,0,0.95}
\definecolor{backcolour}{rgb}{0.95,0.95,0.92}

\lstdefinestyle{mystyle}{
    backgroundcolor=\color{backcolour},   
    commentstyle=\color{codegreen},
    keywordstyle=\color{codeblue},
    numberstyle=\tiny\color{codegray},
    stringstyle=\color{codegreen},
    basicstyle=\ttfamily\footnotesize,
    breakatwhitespace=false,         
    breaklines=true,                 
    captionpos=b,                    
    keepspaces=true,                 
    numbers=left,                    
    numbersep=5pt,                  
    showspaces=false,                
    showstringspaces=false,
    showtabs=false,                  
    tabsize=2
}
 
\lstset{style=mystyle}

\setmarginsrb{3 cm}{2.5 cm}{3 cm}{2.5 cm}{1 cm}{1.5 cm}{1 cm}{1.5 cm}

\title{LAB 7}								
\author{ David Vorous}						
\date{10/19/21}

\makeatletter
\let\thetitle\@title
\let\theauthor\@author
\let\thedate\@date
\makeatother

\pagestyle{fancy}
\fancyhf{}
\rhead{\theauthor}
\lhead{\thetitle}
\cfoot{\thepage}

\begin{document}

\begin{titlepage}
	\centering
    \vspace*{0.5 cm}
\begin{center}    \textsc{\Large   ECE 351 - Section 51 }\\[2.0 cm]	\end{center}
	\textsc{\Large Block Diagrams and System Stability  }\\[0.5 cm]
	\rule{\linewidth}{0.2 mm} \\[0.4 cm]
	{ \huge \bfseries \thetitle}\\
	\rule{\linewidth}{0.2 mm} \\[1.5 cm]
	
	\begin{minipage}{0.4\textwidth}
		\begin{flushleft} \large
			\end{flushleft}
			\end{minipage}~
			\begin{minipage}{0.4\textwidth}
            
			\begin{flushright} \large
			\emph{Submitted By :} \\
			David Vorous  
		\end{flushright}
           
	\end{minipage}\\[2 cm]

\end{titlepage}


\tableofcontents

\pagebreak

\renewcommand{\thesection}{\arabic{section}}

\section{Introduction}

Lab 7 builds on the students current knowledge of Laplace Transforms by introducing Block Diagrams. For our purposes, a Block Diagram represents a transfer function built out of lesser functional blocks. These blocks are typically simpler than their combination, and interact with each other in meaningful ways. This Lab provides a simple feedback block diagram, and a transfer function for each block. The students are tasked with determining and plotting the closed and open loop versions of the transfer function represented by the block diagram. The libraries used in the lab were numpy (imported as "np"), matplotlib (imported as "plt"), and scipy.signals (imported as "sig").

\begin{center}
    \includegraphics[scale=0.333]{Block Diagram}\\
    \text{Block Diagram for ECE 351 Lab 7}
\end{center}

\section{Equations}
Building Block Transfer Functions:
\begin{equation}
    \text{G}(s) = \frac{s + 9}{(s - 8)(s + 2)(s + 4)}
\end{equation}
\begin{equation}
    \text{A}(s) = \frac{s + 4}{(s + 1)(s + 3)}
\end{equation}
\begin{equation}
    \text{B}(s) = (s + 12)(s + 14)
\end{equation}
Open Loop Transfer Function:
\begin{equation}
    \text{OL}(s) = \text{A}(s) \text{G}(s) = \frac{s + 9}{(s - 8)(s + 1)(s + 2)(s + 3)}
\end{equation}
Closed Loop Transfer Function:
\begin{equation}
    \text{CL}(s) = \text{A}(s)\frac{\text{G}(s)}{1+\text{G}(s)\text{B}(s)} = \frac{(s+4)(s+9)}{(s + 1)(s + 3)(2s^{3}+33s^{2}+362s+1448)}
\end{equation}

\pagebreak

\section{Methodology}
Fortunately, entering the transfer functions into the program was not as complicated as the previously mentioned equations make themselves out to be. With the help of the scipy.signal.convolve() function, the pain-staking process of entering the larger forms of OL(s) and CL(s) was avoided to generate each functions set of numerators and denominators. With the numerators and denominators separated, scipy.signals.tf2zpk() could be used to isolate the zeros, poles, and gains for the transfer functions. Afterwards, the OL(s) and CL(s) functions were passed into the scipy.signals.step() function:

\begin{lstlisting}[language=Python]
Gnum = [1,9]
Gden = sig.convolve([1,-8],sig.convolve([1,4],[1,2]))

Anum = [1,4]
Aden = sig.convolve([1,3],[1,1])

Bnum = [1,26,168]

OLnum = [1,9]
OLden = sig.convolve([1,-6,-16],[1,4,3])

CLnum = sig.convolve(Anum, Gnum)
CLden = sig.convolve(Aden, Gden + sig.convolve(Bnum, Gnum))

Gz,Gp,Gk = sig.tf2zpk(Gnum, Gden)
print("G(s) z,p,k = ", Gz, Gp, Gk)

Az,Ap,Ak = sig.tf2zpk(Anum, Aden)
print("\nA(s) z,p,k = ", Az, Ap, Ak)

Bz = np.roots(Bnum)
print("\nB(s) z =     ", Bz)

OLz,OLp,OLk = sig.tf2zpk(OLnum, OLden)
print("\nOL(s) z,p,k = ", OLz, OLp, OLk)

OLstept, OLstepy = sig.step((OLnum, OLden), T = t1)

CLz,CLp,CLk = sig.tf2zpk(CLnum, CLden)
print("\nCL(s) z,p,k = ", CLz, CLp, CLk)

CLstept, CLstepy = sig.step((CLnum, CLden), T = t2)
\end{lstlisting}
Afterwards, the matplotlib.pyplot library was used to plot all relevant data.


\section{Plots}

\begin{center}
    \includegraphics[scale=0.4]{Figure 2021-10-19 155826 (0)}
    \includegraphics[scale=0.4]{Figure 2021-10-19 155826 (1)}
\end{center}

\section{Results}
The results of Lab 7 confirmed the effectiveness of Block Diagrams as well as the theorems which allow their simplification (as opposed to error analysis). The plot of the open loop transfer function confirms it's instability (rapid rise), as it contained at least one positive pole (s = 8). The plot of the closed loop function however, confirms stability, as each of its poles were negative (as could be seen from the sig.tf2zpk() result). In the end, all expectations were therefore met.

\pagebreak

\section{Questions}
\begin{enumerate}
    \item \textit{In Part 1 Task 5, why does convolving the factored terms using scipy.signal.convolve()
result in the expanded form of the numerator and denominator? Would this work with your
user-defined convolution function from Lab 3? Why or why not?}\\\\
    The scipy.signals.convolve() function does not choose to perform either convolutions or polynomial expansion, the process of calculation between the two is simply related. For one, each entry into the input arrays is a time step, whereas with the other, each entry is the coefficient of a polynomial. Knowing these similarities, our hand-written code for convolution should handle polynomial expansion.
    \item \textit{Discuss the difference between the open and closed loop systems from Part 1 and Part 2. How does stability differ for each case, and why?}\\\\
    Referring to the Block Diagram, Open Loop assumes that Block B is not present, where Closed Loop includes it. Unfortunately, without Block B, the system skyrockets, and is therefore unstable. It would seem that Block B tempers this upward acceleration, and allows for the signal to stabilize. We can see the cause more plainly in the factored denominator for OL(s) in Equation 4, when one of the poles is a positive number (s = 8).
    \item \textit{What is the difference between scipy.signal.residue() used in Lab 6 and scipy.signal.tf2zpk() used in this lab?}\\\\
    The residue function returns values for the construction of partial fraction expansions. The tf2zpk function returns values more suited for factoring.
    \item \textit{Is it possible for an open-loop system to be stable? What about for a closed-loop system to be unstable? Explain how or how not for each.}\\\\
    The only requirement for stability is that the real part of all poles be negative. This can of course be done through clever combinations of Blocks in your Block Diagram. Block B for instance, has no denominator... this is no coincidence. Due to the nature of combining feedback blocks, B(s) inevitably ended up in the denominator of the final transfer function, bringing down the stark rise of the result.
\end{enumerate}

\section{Conclusion}
This Lab provided excellent proof of the topics discussed in lecture, and bolstered the methods used to interpret Block Diagrams. While the Block Diagram itself was rather simple, the goal of this lab was less about the applications of Block Diagrams, and more about how to simplify them. Knowing this, the visual feedback of the plots provided a satisfying visualization of a system which could easily exist in the real world. Aside from a hiccup regarding what was meant by the Open Loop version of the Block Diagram, the Lab was clear in it's expectations, and was enjoyable to complete.

All files can be found on my GitHub Repositories tab:\\
\url{https://github.com/David-Vorous?tab=repositories}

\end{document}