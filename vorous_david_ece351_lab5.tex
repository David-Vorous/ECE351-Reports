%%%%%%%%%%%%%%%%%%%%%%%%%%%%%%%%%%%%%%%%%%%
%
%		David Vorous
%		ECE-351-51
%		Lab #5
%		Due: Oct. 5, 2021
%
%%%%%%%%%%%%%%%%%%%%%%%%%%%%%%%%%%%%%%%%%%%

\documentclass[12pt]{report}
\usepackage[english]{babel}
%\usepackage{natbib}
\usepackage{url}
\usepackage[utf8x]{inputenc}
\usepackage{amsmath}
\usepackage{graphicx}
\graphicspath{ {./Graphics/} }
\usepackage{parskip}
\usepackage{fancyhdr}
\usepackage{vmargin}
\usepackage{listings}
\usepackage{hyperref}
\usepackage{xcolor}

\definecolor{codegreen}{rgb}{0,0.6,0}
\definecolor{codegray}{rgb}{0.5,0.5,0.5}
\definecolor{codeblue}{rgb}{0,0,0.95}
\definecolor{backcolour}{rgb}{0.95,0.95,0.92}

\lstdefinestyle{mystyle}{
    backgroundcolor=\color{backcolour},   
    commentstyle=\color{codegreen},
    keywordstyle=\color{codeblue},
    numberstyle=\tiny\color{codegray},
    stringstyle=\color{codegreen},
    basicstyle=\ttfamily\footnotesize,
    breakatwhitespace=false,         
    breaklines=true,                 
    captionpos=b,                    
    keepspaces=true,                 
    numbers=left,                    
    numbersep=5pt,                  
    showspaces=false,                
    showstringspaces=false,
    showtabs=false,                  
    tabsize=2
}
 
\lstset{style=mystyle}

\setmarginsrb{3 cm}{2.5 cm}{3 cm}{2.5 cm}{1 cm}{1.5 cm}{1 cm}{1.5 cm}

\title{LAB 5}								
\author{ David Vorous}						
\date{10/05/21}

\makeatletter
\let\thetitle\@title
\let\theauthor\@author
\let\thedate\@date
\makeatother

\pagestyle{fancy}
\fancyhf{}
\rhead{\theauthor}
\lhead{\thetitle}
\cfoot{\thepage}

\begin{document}

\begin{titlepage}
	\centering
    \vspace*{0.5 cm}
\begin{center}    \textsc{\Large   ECE 351 - Section 51 }\\[2.0 cm]	\end{center}
	\textsc{\Large Step and Impulse Response of an RLC Band-Pass Filter  }\\[0.5 cm]
	\rule{\linewidth}{0.2 mm} \\[0.4 cm]
	{ \huge \bfseries \thetitle}\\
	\rule{\linewidth}{0.2 mm} \\[1.5 cm]
	
	\begin{minipage}{0.4\textwidth}
		\begin{flushleft} \large
			\end{flushleft}
			\end{minipage}~
			\begin{minipage}{0.4\textwidth}
            
			\begin{flushright} \large
			\emph{Submitted By :} \\
			David Vorous  
		\end{flushright}
           
	\end{minipage}\\[2 cm]

\end{titlepage}


\tableofcontents

\pagebreak

\renewcommand{\thesection}{\arabic{section}}

\section{Introduction}

	Lab 5 is meant to introduce the students to even more special library functions, and allow them to compare their hand-calculated results with these library alternatives. Each student is given the task of transforming an RLC bandpass filter into the Laplace domain, solve for H(s), and use inverse Laplace to find the impulse response h(t). After implementing their result into code, they then must compare their plot with the plots generated by the library alternatives. The libraries used in the lab were numpy (imported as "np"), matplotlib (imported as "plt"), and scipy.signals (imported as "sig").

\section{Equations}

Analyzing the circuit in the Laplace domain yielded this function:
\begin{equation}
    \text{H}(s) = \frac{10,000\cdot s}{s^{2} + 10,000\cdot s + \frac {10,000,000}{27}}
\end{equation}
Taking the inverse Laplace of H(s) gave the Impulse Response:
\begin{equation}
    \text{h}(t) = \frac{20,000}{\sqrt[]{373}} \cdot e^{-5,000 \cdot t} \cdot \text{cos} \left( \frac{5,000\cdot \sqrt[]{1,119}}{9} + \text{tan}^{-1} \left( \frac{3\cdot \sqrt[]{1,119}}{373} \right) \right) \cdot u(t)
\end{equation}
Or if exactness isn't necessary:
\begin{equation}
    \text{h}(t) \approx 10,356\cdot e^{-5,000\cdot t}\cdot \text{cos} (18,584\cdot t + 15.059^{\circ}) \cdot u(t)
\end{equation}

Where $u(t)$ is the Unit Step Function.
\pagebreak


\section{Methodology}
Only the function h(t) was explicitly defined in the code, as it was the only calculation to be compared with. H(s) was implemented as a matrix with the first row containing the numerator variable coefficients in descending degree order, and the second row was the same, but involving the denominator instead. The last two functions utilized the scipy.signal library for both a comparison and further data analysis:

\begin{lstlisting}[language=Python]
def reg_h(t):
   return 10355.607 * np.exp(-5000 * t) * np.cos(18584.143 * t + 0.262823)

hs = ([0, 10000, 0],[1, 10000, 370.37037037e6])

impt, impy = sig.impulse(hs, T = t1)
stpt, stpy = sig.step(hs, T = t1)
\end{lstlisting}
Afterwards, the matplotlib.pyplot library was used to plot all functions except H(s).


\section{Plots}
\begin{center}
    \includegraphics[scale=0.3]{Figure 2021-10-05 133121 (0)}
    \includegraphics[scale=0.3]{Figure 2021-10-05 133121 (1)}
\end{center}

\section{Results}

The results of Lab 5 certainly matched expectations; the library impulse response matched my hand-calculated function exactly, and the step response forming a decaying sine-wave is what we'd expect in this situation. One thing worth noting however, is that both of the library functions used in this Lab failed to return the correct plot when the coefficient of the highest degree linear factor in the denominator of H(s) was not 1. Ideally this shouldn't matter, but given this weakness, it's best to factor-out any coefficient that isn't 1, and move on.

\section{Conclusion}

This Lab was short n' sweet, and fulfilled it's purpose in introducing us students to yet another set of useful Library functions. These functions seem a bit more stable than the previously used convolve() function, but perhaps that is to be expected. Aside from the issue with the coefficient of the highest degree linear factor  in the denominator, all seems to work well. The Lab was clear in it's expectations, and was enjoyable to complete.\\

All files can be found on my GitHub Repositories tab:\\
\url{https://github.com/David-Vorous?tab=repositories}

\end{document}